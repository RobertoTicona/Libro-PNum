% --- Definición de colores personalizados ---
\definecolor{backcolour}{rgb}{0.98,0.98,0.98}  % Fondo gris muy claro
\definecolor{codegray}{rgb}{0.5,0.5,0.5}       % Gris para los números
\definecolor{codepurple}{rgb}{0.58,0,0.82}     % Púrpura para strings
\definecolor{keywordblue}{rgb}{0.2,0.2,0.8}    % Azul para palabras clave
\definecolor{commentgreen}{rgb}{0,0.5,0}       % Verde para comentarios

% --- Configuración general de listings ---
\lstset{
	backgroundcolor=\color{backcolour},        % Fondo claro
	basicstyle=\ttfamily\footnotesize,         % Fuente monoespaciada pequeña
	breaklines=true,                           % Permitir cortes de línea
	captionpos=b,                              % Título debajo
	keepspaces=true,                           % Mantener espacios
	numbers=left,                              % Números a la izquierda
	numbersep=5pt,                             % Espacio de margen para números
	numberstyle=\tiny\color{codegray},         % Color de los números
	showspaces=false,
	showstringspaces=false,
	showtabs=false,
	tabsize=4,
	frame=single,                              % Marco alrededor del código
	rulecolor=\color{gray!70},                 % Color del marco
	keywordstyle=\color{keywordblue}\bfseries, % Palabras clave azules en negrita
	commentstyle=\color{commentgreen}\itshape, % Comentarios verdes en cursiva
	stringstyle=\color{codepurple},            % Strings púrpuras
	postbreak=\mbox{\textcolor{red}{$\hookrightarrow$}\space},
	linewidth=\textwidth                       % Ajuste al ancho del texto
}

% ==============================
% ENTORNOS PERSONALIZADOS
% ==============================

% --- Entorno para Python ---
\lstnewenvironment{pythoncode}[1][]
{\lstset{language=Python, captionpos=b, #1}}{}

% --- Entorno para R ---
\lstnewenvironment{rcode}[1][]
{\lstset{language=R, captionpos=b, #1}}{}