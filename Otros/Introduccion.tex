\chapter*{Introducción}
\addcontentsline{toc}{chapter}{Introducción}

La programación numérica es una rama fundamental dentro de las ciencias computacionales y aplicadas, cuyo objetivo es proporcionar métodos y herramientas para resolver problemas matemáticos mediante el uso de algoritmos y programas informáticos. A través de ella, es posible aproximar soluciones a ecuaciones que no pueden resolverse de forma analítica, optimizar funciones complejas, y analizar sistemas dinámicos en diversas áreas del conocimiento.

El propósito de este libro es introducir al lector en los principios y aplicaciones prácticas de la programación numérica, utilizando lenguajes de programación de propósito científico como \texttt{Python} y \texttt{R}. A lo largo de los capítulos, se desarrollan los fundamentos teóricos y computacionales necesarios para comprender los métodos numéricos más empleados en ingeniería, física, economía y nutrición, entre otros campos.

En la \textbf{primera unidad}, se abordarán los conceptos básicos de la programación numérica, la definición de funciones matemáticas y sus restricciones, así como los métodos clásicos para \textit{encontrar raíces de ecuaciones no lineales}, tales como el método de bisección, el método de Newton-Raphson y el método de la secante. Estos métodos serán implementados paso a paso en código, permitiendo observar su comportamiento, eficiencia y convergencia.

En la \textbf{segunda unidad}, se explorarán métodos más avanzados, entre ellos el \textit{gradiente descendente} y sus aplicaciones en optimización, la \textit{interpolación polinómica} como técnica para aproximar funciones a partir de datos discretos, y la \textit{diferenciación numérica}, que permite estimar derivadas de funciones cuando no se dispone de una expresión analítica. Cada tema será acompañado de ejemplos prácticos y ejercicios que ayudarán a reforzar la comprensión teórica mediante la experimentación computacional.

Además, se presentarán recomendaciones sobre buenas prácticas en la escritura de código científico, el uso eficiente de librerías numéricas, y la validación de resultados mediante análisis de error. El enfoque de este libro combina la precisión matemática con la aplicación práctica, fomentando el desarrollo de habilidades analíticas y computacionales en el lector.

En conjunto, este material busca no solo enseñar los fundamentos de la programación numérica, sino también promover un pensamiento crítico y estructurado al enfrentar problemas reales que requieren soluciones computacionales.